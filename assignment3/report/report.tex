\documentclass{article}

% if you need to pass options to natbib, use, e.g.:
% \PassOptionsToPackage{numbers, compress}{natbib}
% before loading nips_2016
%
% to avoid loading the natbib package, add option nonatbib:
% \usepackage[nonatbib]{nips_2016}

\usepackage{nips_2016}

% to compile a camera-ready version, add the [final] option, e.g.:
% \usepackage[final]{nips_2016}

\usepackage[utf8]{inputenc} % allow utf-8 input
\usepackage[T1]{fontenc}    % use 8-bit T1 fonts
\usepackage{hyperref}       % hyperlinks
\usepackage{url}            % simple URL typesetting
\usepackage{booktabs}       % professional-quality tables
\usepackage{amsfonts}       % blackboard math symbols
\usepackage{nicefrac}       % compact symbols for 1/2, etc.
\usepackage{microtype}      % microtypography
\usepackage{amsmath}
\usepackage{mathtools}
\title{Deep Learning Assignment 3}

% The \author macro works with any number of authors. There are two
% commands used to separate the names and addresses of multiple
% authors: \And and \AND.
%
% Using \And between authors leaves it to LaTeX to determine where to
% break the lines. Using \AND forces a line break at that point. So,
% if LaTeX puts 3 of 4 authors names on the first line, and the last
% on the second line, try using \AND instead of \And before the third
% author name.
\author{
  Peter Yun-shao Sung
  \texttt{yss265@nyu.edu} \\
  %% examples of more authors
  %% \And
  %% Coauthor \\
  %% Affiliation \\
  %% Address \\
  %% \texttt{email} \\
  %% \AND
  %% Coauthor \\
  %% Affiliation \\
  %% Address \\
  %% \texttt{email} \\
  %% \And
  %% Coauthor \\
  %% Affiliation \\
  %% Address \\
  %% \texttt{email} \\
  %% \And
  %% Coauthor \\
  %% Affiliation \\
  %% Address \\
  %% \texttt{email} \\
}

\begin{document}
% \nipsfinalcopy is no longer used

\maketitle
\section{General Questions}

\section{Softmax regression gradient calculation}
Given
\begin{equation}
\hat{y} = \sigma (Wx+b) \textbf{ , where $x \in \mathbb{R}^d$,$W \in \mathbb{R}^{k\times d}$, $b \in \mathbb{R}^k$}
\end{equation}
where d is the input dimension, k is the number of classes, $\sigma$ is the softmax function:
\begin{equation}
\sigma(a)_i = {exp(a_i) \over \sum_j exp(a_j)}
\end{equation}
Which means a given input $x$ will output $y$ with probability of each class

\subsection{Derive ${\partial l\over \partial W_{ij}}$}
If the given cross-entropy loss defined as followed:
\begin{equation}
l(y, \hat{y}) = -\sum_i y_i\log\hat{y_i}
\end{equation}
As $W_{ij}$ will affect the prediction of class $i$ by multipling index $j$ in $x$, therefore we can derive:
\begin{equation}
{\partial l\over \partial W_{ij}} = {\partial l \over \partial \hat{y_i}} {\partial \hat{y_i} \over \partial W_{ij}}
\end{equation}
where:
\begin{equation}
l(y, \hat{y}) = -\sum_i y_i\log\hat{y_i} = -(y_i\log\hat{y_1} + y_2\log\hat{y_2} + \dots + y_i\log\hat{y_i} + \dots)
\end{equation}
and therefore
\begin{equation}
{\partial l \over \partial \hat{y_i}} = {-y_i\over \hat{y_i}}
\end{equation}
And we can rewrite (1) and (2) and care the value only for $\hat y_i$:
\begin{equation}
\hat{y_i} = {exp(a_i) \over \Sigma_j exp(a_j)} = {exp(a_i) \over C + exp(a_i)} \textbf{,  where $C = \sum_{k\neq i} exp(a_k)$}
\end{equation}
Since
\begin{equation}
{\partial exp(a_i) \over \partial W_{ij}} = W_{ij}exp(a_i)
\end{equation}
Therefore
\begin{equation}
{\partial \hat{y_i} \over \partial W_{ij}} = W_{ij}\hat{y_i}(1-\hat{y_i})
\end{equation}
% \begin{equation}
% \hat{y_i} = {e^{(W_{ij}X_j+b_i)} \over C + e^{(W_{ij}X_j+b_i)} } \textbf{, where $C = \sum_{k\neq j} e^{W_{ik}X_k+b_i}$}
% \end{equation}
% \begin{equation}
% {\partial \hat{y_i} \over \partial W_{ij}} = {X_je^{(W_{ij}X_j+b_i)} \over C + e^{(W_{ij}X_j+b_i)} } - {X_je^{2(W_{ij}X_j+b_i)} \over (C + e^{(W_{ij}X_j+b_i)})^2} = X_j \hat{y_i} (1- \hat{y_i})
% \end{equation}
Combining (6) and (9) to (4), and we will get the result:
\begin{equation}
{\partial l\over \partial W_{ij}} = {\partial l \over \partial \hat{y_i}} {\partial \hat{y_i} \over \partial W_{ij}} = -X_jy_i(1- \hat{y_i})
\end{equation}
\subsection{What happen when $y_{c_1}=1, \hat{y}_{c_2}=1, c_1 \neq c_2$}
This means something like $y = [1, 0, 0]^T$ and $\hat{y} = [0,0,1]^T$, and the predict is far different from true lable.
\section{Chain rule}
\section{Variants of pooling}
\section{Convolution}
(a) As it is using 3x3 kernal along x and y axis of input, which is 5 and 5 respectively. The output of this layer will be $(5-3+1)\times(5-3+1)$ which is 3x3.\\
(b) Assuming the kernel operation is point-point multiplication and summation, then the output of this layer is:\\
$\begin{pmatrix}
  109 & 92 & 72 \\[0.4em]
  108 & 85 & 74 \\[0.4em]
  110 & 74 & 79
\end{pmatrix}$ \\
(c)
$\begin{pmatrix}
  4 & 7 & 10 & 6 & 3 \\[0.4em]
  9 & 17 & 25 & 16 & 8 \\[0.4em]
  11 & 23 & 34 & 23 & 11 \\[0.4em]
  7 & 16 & 24 & 17 & 8 \\[0.4em]
  2 & 6 & 9 &7 & 3
\end{pmatrix}$ \\


\section{Optimization}
\section{Top-k error}
\section{t-SNE}
\section{Proximal gradient descent}


\end{document}






